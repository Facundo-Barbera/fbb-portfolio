%! Author = Facundo Bautista Barbera
%! Date = 9/16/2025

% Preamble
\documentclass[11pt]{article}

% Packages
\usepackage[T1]{fontenc}
\usepackage[utf8]{inputenc}
\usepackage[spanish]{babel}
\usepackage{amsmath}
\usepackage{graphicx}
\usepackage{float}
\usepackage{microtype}
\PassOptionsToPackage{hyphens}{url}
\usepackage{hyperref}
\hypersetup{breaklinks=true}
\setlength{\emergencystretch}{3em}

% Document
\begin{document}

% Title
\title{
	Tarea: Investigaci\'on sobre el uso de la IA en tu Carrera Profesional
}
\author{
	Facundo Bautista Barbera A01066843
}
\date{\today}
\maketitle

\section*{Introducci\'on}

Estoy estudiando Ingenier\'ia en Ciencia de Datos, que b\'asicamente se trata de crear sistemas que puedan analizar grandes cantidades de datos y sacar informaci\'on \'util de ellos. En mi carrera vamos a dise\~nar pipelines de datos, crear modelos que puedan predecir cosas y desarrollar sistemas de machine learning para ayudar a las empresas a tomar mejores decisiones. En esta tarea voy a investigar c\'omo la inteligencia artificial puede ayudarnos a ser m\'as productivos en nuestro trabajo como futuros cient\'ificos de datos.
Este documento es un boceto de una actividad mas completa, por lo que no debe ser tomada una versión final.

\section*{Casos de uso de la IA en mi carrera}

La IA en ciencia de datos es algo que vemos de cerca en clases y proyectos. Para esta investigaci\'on escog\'i cuatro \textbf{casos de uso principales} que me parecieron interesantes para mostrar c\'omo la IA puede ayudarnos a ser m\'as eficientes en nuestro trabajo.

\begin{enumerate}
	\item \textbf{Optimizaci\'on en Ciencia de Datos:} La IA contribuye en problemas de optimizaci\'on matem\'atica, heur\'isticas y optimizaci\'on de hiperpar\'ametros mediante sistemas AutoML.

	\item \textbf{An\'alisis Multivariado:} Algoritmos de IA potencian la reducci\'on de dimensionalidad, la clasificaci\'on y la regresi\'on. T\'ecnicas como autoencoders variacionales y t-SNE permiten generar representaciones m\'as interpretables y robustas de datos complejos, facilitando el descubrimiento de patrones ocultos.

	\item \textbf{Topolog\'ia y Geometr\'ia de Datos:} La IA permite descubrir estructuras ocultas en conjuntos de datos complejos mediante el an\'alisis topol\'ogico de datos (TDA). Aplicando conceptos de homolog\'ia persistente y m\'etricas topol\'ogicas, se pueden identificar caracter\'isticas invariantes que son robustas al ruido y perturbaciones.

	\item \textbf{Simulaci\'on y Modelado:} La IA apoya en la simulaci\'on de sistemas complejos y en la generaci\'on de escenarios hipot\'eticos. Los modelos generativos como GANs y modelos de difusi\'on facilitan la creaci\'on de datos sint\'eticos para pruebas y validaci\'on, mejorando la toma de decisiones estrat\'egicas.
\end{enumerate}

\section*{Prompt General}
Utilizando prompt engineering, se dise\~n\'o un prompt base para todos los casos de uso:

\begin{quote}
	\textit{``Act\'ua como un experto en ciencia de datos. Con base en el siguiente tema: [TEMA ESPEC\'IFICO], investiga fuentes acad\'emicas recientes (2023-2025) y genera una presentaci\'on ejecutiva que explique: 1) Fundamentos te\'oricos del concepto, 2) Aplicaciones pr\'acticas con ejemplos reales de empresas, 3) M\'etricas cuantitativas de mejora en productividad (porcentajes, tiempos reducidos, ROI), 4) Limitaciones actuales y consideraciones \'eticas, 5) Tendencias futuras. Incluye referencias verificables y c\'odigo ejemplo cuando sea relevante.''}
\end{quote}

\section*{Metodolog\'ia Detallada}

Para cada caso de uso se sigui\'o un proceso sistem\'atico:

Para conseguir la informaci\'on us\'e varias herramientas de IA como ChatGPT-5, Claude Opus 4.1, Gemini Pro 2.5, Microsoft Copilot y Meta AI (modelo local usando ML Studio, no logró hacer la presentación).
Fui mejorando mis prompts poco a poco seg\'un las respuestas que me daban, aplicando t\'ecnicas de prompt engineering y también con ayuda de otras inteligencias artificiales.
Tambi\'en tuve que verificar que las referencias que me suger\'ian fueran reales buscando directamente en bases de datos acad\'emicas, porque a veces las IA se inventan citas.
(Nota: Logre hacer que pr\'acticamente todas las referencias que me escribieron fueran reales.)

Despu\'es compar\'e las respuestas de las diferentes IA para ver qu\'e ten\'ian en com\'un y en qu\'e se diferenciaban.
Tuve que editar y mejorar un poco el contenido que generaron, corrigiendo errores y haciendo que todo tuviera m\'as sentido.
Al final decidi entregar la presentación sin muchas modificaciones ya que no estaba seguro si era necesario modificar mucho para esta primera entrega.

\section*{Resultados y Hallazgos Principales}

\subsection*{Impacto en la Productividad}

En mi investigaci\'on encontr\'e que la IA realmente puede ayudarnos mucho a ser m\'as productivos en ciencia de datos.
Creo que en particular (fuera de lo que encontre) también puedo decir que el uso mas particular para le inteligencia artificial que tengo como cientifico de datos es que me ayuda con la programación.

Otra cosa interesante es que la optimizaci\'on bayesiana con estrategias evolutivas funciona mejor que los m\'etodos tradicionales para buscar hiperpar\'ametros.
Generalmente para este tipo de pruebas se hacen procesos como el grid search y también validaciones cruzadas.
La IA puede ayudar a automatizar la decisión de cu\'ales hiperpar\'ametros vamos a probar.
Esto significa que no tenemos que ser expertos en todo desde el principio.

\section*{Conclusiones}

Despu\'es de hacer toda esta investigaci\'on, creo que el uso de IA en ciencia de datos realmente a cambiado mucho la forma en que trabajamos.
Lo que m\'as se puede destacar es c\'omo la IA puede ayudarnos a ser mucho m\'as productivos automatizando tareas complejas y repetitivas, lo que nos deja tiempo para enfocarnos en la soluci\'on de los problemas.

Pero tambi\'en me di cuenta de que es muy importante mantener un balance entre automatizaci\'on y supervisi\'on humana.
La IA es una herramienta muy poderosa, pero siempre necesitamos validar constantemente los resultados y considerar aspectos \'eticos.
Creo que el futuro va a estar marcado por la colaboraci\'on entre humanos y la IA, donde cada uno aporta lo que sabe hacer mejor.

En general, la inteligencia artificial esta siendo una herramienta que tiene un crecimiento constante (en todos los aspectos) y no podemos detenerlo.
Es mejor adaptarnos a utilizarla, sin quitar la parte del trabajo humano.

\section*{Referencias}

\begin{enumerate}
	\item Hastie, T., Tibshirani, R., \& Friedman, J. (2009). \textit{The Elements of Statistical Learning: Data Mining, Inference, and Prediction} (2nd ed.). Springer. Disponible en: \url{https://hastie.su.domains/ElemStatLearn/}

	\item Bronstein, M., Bruna, J., Cohen, T., \& Veli\v{c}kovi\'c, P. (2021). ``Geometric Deep Learning: Grids, Groups, Graphs, Geodesics, and Gauges.'' \textit{arXiv preprint arXiv:2104.13478}. Disponible en: \url{https://arxiv.org/abs/2104.13478}

	\item Goodfellow, I., Bengio, Y., \& Courville, A. (2016). \textit{Deep Learning}. MIT Press. Disponible en: \url{https://www.deeplearningbook.org/}

	\item Vincent, A.M., \& Jidesh, P. (2023). ``An improved hyperparameter optimization framework for AutoML systems using evolutionary algorithms.'' \textit{Scientific Reports}, 13, 4737. \url{https://doi.org/10.1038/s41598-023-32027-3}

	\item Bischl, B., Binder, M., Lang, M., et al. (2023). ``Hyperparameter optimization: Foundations, algorithms, best practices, and open challenges.'' \textit{WIREs Data Mining and Knowledge Discovery}, 13(2), e1484. \url{https://doi.org/10.1002/widm.1484}

	\item Otter, N., Porter, M. A., Tillmann, U., Grindrod, P., \& Harrington, H. A. (2017). ``A roadmap for the computation of persistent homology.'' \textit{EPJ Data Science}, 6, 17. \url{https://doi.org/10.1140/epjds/s13688-017-0109-5}

	\item Chazal, F., \& Michel, B. (2021). ``An Introduction to Topological Data Analysis: Fundamental and Practical Aspects for Data Scientists.'' \textit{Frontiers in Artificial Intelligence}, 4, 667963. \url{https://doi.org/10.3389/frai.2021.667963}

	\item Davenport, T. H., \& Bean, R. (2024). ``Five Key Trends in AI and Data Science for 2024.'' \textit{MIT Sloan Management Review}. Disponible en: \url{https://sloanreview.mit.edu/article/five-key-trends-in-ai-and-data-science-for-2024/}

	\item Dell'Acqua, F., et al. (2023). ``Navigating the Jagged Technological Frontier: Field Experimental Evidence of the Effects of AI on Knowledge Worker Productivity and Quality.'' \textit{Harvard Business School Working Paper}. Disponible en: \url{https://www.hbs.edu/faculty/Pages/item.aspx?num=64700}

	\item Bick, A., Blandin, A., \& Deming, D. (2025). ``The Rapid Adoption of Generative AI.'' \textit{Federal Reserve Bank of St. Louis Working Paper 2024-027C}. Disponible en: \url{https://doi.org/10.20955/wp.2024.027}

	\item McKinsey Global Institute. (2023). ``The economic potential of generative AI: The next productivity frontier.'' McKinsey \& Company. Disponible en: \url{https://www.mckinsey.com/capabilities/mckinsey-digital/our-insights/}

	\item Salehin, I., Islam, M. S., Saha, P., et al. (2024). ``AutoML: A systematic review on automated machine learning with neural architecture search.'' \textit{Journal of Information and Intelligence}, 2(1), 52-81. \url{https://doi.org/10.1016/j.jiixd.2023.10.002}

	\item Su, Z., Liu, X., Hamdan, L. B., et al. (2025). ``Topological Data Analysis and Topological Deep Learning Beyond Persistent Homology--A Review.'' \textit{arXiv preprint arXiv:2507.19504}. Disponible en: \url{https://arxiv.org/abs/2507.19504}

	\item Desai, M. (2024). ``Introducing Code-First AutoML and Hyperparameter Tuning: Now in Public Preview for Fabric Data Science.'' \textit{Microsoft Fabric Blog}. Disponible en: \url{https://blog.fabric.microsoft.com/en-us/blog/introducing-code-first-automl}

	\item Carlsson, G. (2009). ``Topology and data.'' \textit{Bulletin of the American Mathematical Society}, 46(2), 255-308. \url{https://doi.org/10.1090/S0273-0979-09-01249-X}
\end{enumerate}


\end{document}
