%! Author: Facundo Bautista Barbera
%! Date: 2025-09-26
%! TEX root = main.tex

\documentclass{article}

\usepackage[utf8]{inputenc}
\usepackage[T1]{fontenc}
\usepackage[spanish]{babel}
\usepackage{amsmath}

\title{Capítulo 1}
\author{Facundo Bautista Barbera}
\date{\today}

\begin{document}
\maketitle

\section*{Problema}

Alden Construction está compitiendo contra Forbes Construction en una licitación para un proyecto. Alden considera que la oferta de Forbes, denotada por $B$, es una variable aleatoria con la siguiente función de probabilidad:

$$
	P(B = 6000) = 0.40,\quad
	P(B = 8000) = 0.30,\quad
	P(B = 11000) = 0.30.
$$

Se sabe que el costo de Alden para completar el proyecto es de \$6{,}000. Se pide
utilizar los criterios de decisión vistos en clase (maximin, maximax, minimax con
arrepentimiento y valor esperado) para determinar la oferta que debe presentar Alden.

En caso de empate entre las ofertas, Alden gana la licitación.

\textbf{Sugerencia:} Sea $p$ la oferta de Alden. Considere los siguientes intervalos:

\[
	p \leq 6000,\quad
	6000 < p \leq 8000,\quad
	8000 < p \leq 11000,\quad
	p > 11000,
\]

y determine en cada caso la ganancia de Alden en función de su propia oferta $p$ y de la oferta de Forbes $B$.

\newpage

Tabla de utilidades

\begin{tabular}{c|ccc}
	\text{Oferta de Alden } (p) & B=6000 & B=8000 & B=11000 \\
	\hline
	5000                        & -1000  & -1000  & -1000   \\
	6000                        & 0      & 0      & 0       \\
	7000                        & 0      & 1000   & 1000    \\
	8000                        & 0      & 2000   & 2000    \\
	9000                        & 0      & 0      & 3000    \\
	10000                       & 0      & 0      & 4000    \\
	11000                       & 0      & 0      & 5000    \\
	12000                       & 0      & 0      & 0       \\
\end{tabular}

Tabla de arrepentimiento

\begin{tabular}{c|ccc}
	\text{Oferta de Alden } (p) & B=6000 & B=8000 & B=11000 \\
	\hline
	5000                        & 1000   & 3000   & 6000    \\
	6000                        & 0      & 2000   & 5000    \\
	7000                        & 0      & 1000   & 4000    \\
	8000                        & 0      & 0      & 3000    \\
	9000                        & 0      & 2000   & 2000    \\
	10000                       & 0      & 2000   & 2000    \\
	11000                       & 0      & 2000   & 2000    \\
	12000                       & 0      & 2000   & 5000    \\
\end{tabular}

Valor esperado

\begin{tabular}{c|c}
	\text{Oferta de Alden } (p) & \text{Valor esperado} \\
	\hline
	5000                        & -1000                 \\
	6000                        & 0                     \\
	7000                        & 600                   \\
	8000                        & 1200                  \\
	9000                        & 900                   \\
	10000                       & 1200                  \\
	11000                       & 1500                  \\
	12000                       & 0                     \\
\end{tabular}

\subsection*{Resultados}

Para la tabla de utilidades:
\begin{itemize}
	\item Para maximin, el valor maximo de los minimos es $0$ para una oferta de $11000$.
	\item Para maximas, el valor maximo de los maximos es $5000$ para una oferta de $11000$.
	\item Para minimax con la tabla de arrepentimiento, el valor es $2000$ para una oferta $11000$.
	\item Para el valor esperado, el valor es de $1500$ para una oferta de $11000$.
\end{itemize}
\end{document}
