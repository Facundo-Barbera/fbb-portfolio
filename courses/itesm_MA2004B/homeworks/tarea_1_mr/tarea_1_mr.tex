%! Author: Facundo Bautista Barbera A01066843
%! Date: 2025-09-25
%! TEX root = tarea_1_mr.tex

\documentclass{article}
\usepackage{amsmath}
\usepackage{amssymb}
\usepackage{bm}
\usepackage{graphicx}

\title{Tarea 1 MR}
\author{Facundo Bautista Barbera A01066843}
\date{\today}

\begin{document}
\maketitle

\section*{Planteamiento del problema}

\begin{quote}
	\itshape
	Una patrulla policiaca vigila un vecindario conocido por sus actividades pandilleriles.
	Durante un patrullaje hay 60\% de probabilidades  de llegar a tiempo al lugar  donde se requiere ayuda; si no sucede algo, continuará el patrullaje regular.
	Después de recibir una llamada, hay 10\% de probabilidad de cancelación (en cuyo caso el patrullaje normal se reanuda), y 30\% de probabilidad de que la unidad ya esté respondiendo a la llamada anterior.
	Cuando la patrulla llega a la escena del suceso, hay 10\% de probabilidades de que los instigadores hayan desaparecido (en cuyo caso reanuda su patrullaje), y 40\% de probabilidades de que se haga una aprehensión de inmediato.
	De otro modo, los oficiales rastrearán el área.
	Si ocurre una aprehensión , hay 60\% de probabilidades de trasladar a los sospechosos a la estación de policía, de lo contrario son liberados y la unidad regresa a patrullar.
\end{quote}

\section*{Estados y sus probabilidades}

Para empezar este problema cuenta con 5 estados:
\begin{itemize}
	\item E1, Patrullaje: Se podría considerar como un estado inicial. La patrulla siempre vuelve a este estado, por lo que podemos decir que es recurrente.
	\item E2, Respondiendo: Este estado aparenta ser transitorio, pero se puede clasificar como recurrente también. No importa el estado de la cadena, siempre se volverá a este paso.
	\item E3, En escena: Otro estado recurrente, siempre se vuelve a este estado en algún punto.
	\item E4, Rastreo: Estado transitorio, que va a al estado de patrullaje o al estado de Traslado.
	\item E5, Traslado: Transitorio, siempre regresa al estado de patrullaje,
\end{itemize}

\section*{Matriz de transición}
\[
	P =
	\begin{array}{c|ccccc}
		   & E1  & E2  & E3  & E4  & E5  \\
		\hline
		E1 & 0.4 & 0.6 & 0   & 0   & 0   \\
		E2 & 0.1 & 0.3 & 0.6 & 0   & 0   \\
		E3 & 0.1 & 0   & 0   & 0.5 & 0.4 \\
		E4 & 0.4 & 0   & 0   & 0   & 0.6 \\
		E5 & 1   & 0   & 0   & 0   & 0
	\end{array}
\]

\section*{Preguntas}

\subsection*{Si la patrulla se encuentra en este momento en la escena de una llamada, determine la probabilidad de que haga una aprehensión en dos patrullajes.}

La única forma de llegar a E5 desde E2 en dos pasos, es específicamente pasando por E4.
Por lo que calculamos la probabilidad conjunta de esto:

$$
	E3 \rightarrow E4 \rightarrow E5 = 0.5 \times 0.6 = 0.30
$$

La probabilidad de llegar a E5 en dos pasos estando en E3 es de del 30\%.

\subsection*{Si la patrulla se encuentra en este momento en su patrullaje regular, determine la probabilidad de que haga una aprehensión en 10 patrullajes. Utilice una calculadora en línea para realizar los cálculos.}

Idealmente lo que se busca es que, a partir del estado $E1$, calculemos la probabilidad de estar en $E5$ en exactamente 10 pasos, es decir $t=10$.

Se puede usar la siguiente expresión junto con la matriz de transición:

$$
	\mathbb{P}(X_{10}=E5 \mid X_0=E1) = (P^{10})_{E1,E5}.
$$

Usando una calculadora en línea se obtiene:

$$
	(P^{10})_{E1,E5} \approx 0.1261.
$$

\newpage
\section*{Simulación}

Para la ultima pregunta:
\begin{quote}
	\itshape
	Si la patrulla hace una aprehensión, realiza una simulación para encontrar la probabilidad de que, a largo plazo, la patrulla traslade a los sospechosos a la estación de policía.
\end{quote}

Se creo un código de python que devolvió la siguiente gráfica:

\begin{figure}[h]
	\centering
	\includegraphics[width=0.8\textwidth]{Figure_1.png}
	\caption{Simulación de la probabilidad a largo plazo}
	\label{fig:simulation}
\end{figure}

La probabilidad final calculada (aproximada) es de $0.6$

\end{document}

