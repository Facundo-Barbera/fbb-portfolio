% !TEX program = pdflatex
% !TEX options = -aux-directory=out -output-directory=out -synctex=1 -interaction=nonstopmode -file-line-error "%DOC"
\documentclass{article}
\usepackage{amsmath}

\begin{document}

\section*{Actividad}
Los clientes pueden ser leales a marcas de productos pero pueden ser persuadidos mediante publicidad y mercadotecnia inteligentes para que cambien de marcas. Considere el caso de 3 marcas: A, B y C. Los clientes que se “mantienen” leales a una marca dada se estiman en 75\%, con un margen de solo el 25\% para que sus competidores hagan un cambio. Los competidores lanzan sus campañas publicitarias una vez al año. Para los clientes de la marca A, las probabilidades de que cambien a las marcas B y C son de 0.1 y 0.15, respectivamente. Los clientes de la marca B son propensos a cambiar a las marcas A y C con probabilidades iguales a 0.2 y 0.05, respectivamente. Los clientes de la marca C pueden cambiar a la marca A y B con probabilidades iguales.

\section*{Matriz de transición}

$$
	P =
	\begin{array}{c|ccc}
		  & A     & B     & C    \\
		\hline
		A & 0.75  & 0.1   & 0.15 \\
		B & 0.2   & 0.75  & 0.05 \\
		C & 0.125 & 0.125 & 0.75 \\
	\end{array}
$$
\section*{Estado estable}

Para obtener el estado estable usamos $\pi = \pi P$:

$$
	(\pi_1, \pi_2, \pi_3) = (
	0.75\pi_1 + 0.2\pi_2 + 0.125\pi_3,
	0.1\pi_1 +0.75\pi_2 + 0.125\pi_3,
	0.15\pi_1 + 0.05\pi_2 + 0.75\pi_3
	)
$$

Lo que nos deja con el siguiente sistema de ecuaciones:

$$
	\begin{aligned}
		-0.25\pi_1 + 0.2\pi_2 + 0.125\pi_3 & = 0 \\
		0.1\pi_1 - 0.25\pi_2 + 0.125\pi_3  & = 0 \\
		0.15\pi_1 + 0.05\pi_2 - 0.25\pi_3  & = 0 \\
		\pi_1 + \pi_2 + \pi_3              & = 1
	\end{aligned}
$$
Resolviendo este sistema de ecuaciones con una calculadora de matrices obtenemos:

$$
	\pi_1 = \frac{15}{38}, \quad \pi_2 = \frac{35}{114}, \quad \pi_3 = \frac{17}{57}
$$

\end{document}
