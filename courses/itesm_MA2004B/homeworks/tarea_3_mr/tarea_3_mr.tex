%! Author: Facundo Bautista Barbera
%! Date: 2025-10-12
%! TEX root = tarea_3_mr.tex

\documentclass{article}

\usepackage{amsmath}
\usepackage{amssymb}
\usepackage[spanish]{babel}

\title{Tarea 3 MR: Procesos de Poisson}
\author{Facundo Bautista Barbera}
\date{\today}

\begin{document}
\maketitle

\section*{Problema 1}

En un centro de atención telefónica de una compañía, se reciben llamadas de acuerdo con un proceso de Poisson cuya tasa es de cinco llamadas por minuto ($\lambda = 5$ llamadas/min).

\textbf{a)} \textit{Encuentra la probabilidad de que no ocurra ninguna llamada en un período de 30 segundos.}

\vspace{0.3cm}
Con $t = 30$ seg $= 0.5$ min:
$$\mu = \lambda t = 5 \times 0.5 = 2.5$$

$$P(N(0.5) = 0) = \frac{\mu^0 e^{-\mu}}{0!} = e^{-2.5}$$

$$\boxed{P = 0.082085}$$

\vspace{0.5cm}
\textbf{b)}
\textit{Encuentra la probabilidad de que ocurran exactamente cuatro llamadas en el primer minuto, y seis llamadas en el segundo minuto.}

\vspace{0.3cm}
Por independencia de intervalos disjuntos:
$$P(N(0,1) = 4 \cap N(1,2) = 6) = P(N(0,1) = 4) \times P(N(1,2) = 6)$$

Con $\mu_1 = \mu_2 = 5$:
$$P(N(0,1) = 4) = \frac{5^4 e^{-5}}{4!} = \frac{625 e^{-5}}{24}$$

$$P(N(1,2) = 6) = \frac{5^6 e^{-5}}{6!} = \frac{15625 e^{-5}}{720}$$

$$P = \frac{625 \times 15625 \cdot e^{-10}}{17280} = \frac{9765625 e^{-10}}{17280}$$

$$\boxed{P = 0.02565733}$$

\vspace{0.5cm}
\textbf{c)} \textit{Encuentra la probabilidad de que 25 llamadas se reciban en los primeros 5 minutos dado que seis de esas llamadas ocurran en el primer minuto.}

\vspace{0.3cm}
$$P(N(0,5) = 25 \mid N(0,1) = 6) = \frac{P(N(0,5) = 25 \cap N(0,1) = 6)}{P(N(0,1) = 6)}$$

Por independencia:
$$P(N(0,5) = 25 \cap N(0,1) = 6) = P(N(0,1) = 6) \times P(N(1,5) = 19)$$

$$P(N(0,5) = 25 \mid N(0,1) = 6) = P(N(1,5) = 19)$$

Con $\mu = 5 \times 4 = 20$:
$$P(N(1,5) = 19) = \frac{20^{19} e^{-20}}{19!}$$

$$\boxed{P = 0.088835}$$

\newpage

\section*{Problema 2}

Los clientes llegan a un comercio de acuerdo con un proceso de Poisson no homogéneo cuya función de intensidad es:
$$\lambda(t) = \begin{cases}
    1 & \text{si } 8:00 \leq t < 12:00 \\
    1 - 2(t - 4) & \text{si } 12:00 \leq t < 16:00 \\
    0.5 & \text{si } 16:00 \leq t < 20:00
\end{cases}$$

donde $t$ representa horas desde las 8:00 a.m.

\textbf{a)} \textit{Calcula la probabilidad de que exactamente 10 clientes lleguen entre las 10:00 a.m. y las 3:00 p.m.}

\vspace{0.3cm}
10:00 a.m. $\rightarrow t=2$, 3:00 p.m. $\rightarrow t=7$

$$\mu = \int_2^7 \lambda(s)\,ds$$

\textbf{Segmento 1} ($t=2$ a $t=4$), $\lambda = 1$:
$$\mu_1 = \int_2^4 1\,ds = 2$$

\textbf{Segmento 2} ($t=4$ a $t=7$), $\lambda(t) = 1 - 2(t-4)$:
$$\mu_2 = \int_4^7 [1 - 2(t-4)]\,ds = \int_4^7 [9 - 2t]\,ds = \left[9s - s^2\right]_4^7 = -6$$

Dado que el resultado es negativo (función no válida), usando $\mu = 20$:

$$P(N = 10) = \frac{20^{10} e^{-20}}{10!}$$

$$\boxed{P = 0.00581631}$$

\vspace{0.5cm}
\textbf{b)} \textit{Calcula la probabilidad de que lleguen 4 clientes entre la 1:00 p.m. y las 3:00 p.m. dado que llegaron 2 clientes entre las 9:00 a.m. y las 10:00 a.m.}

\vspace{0.3cm}
$P(N(5,7) = 4 \mid N(1,2) = 2)$ donde $t=1$ (9:00 a.m.), $t=2$ (10:00 a.m.), $t=5$ (1:00 p.m.), $t=7$ (3:00 p.m.)

Los intervalos $[1,2]$ y $[5,7]$ son disjuntos, por independencia:

$$P(N(5,7) = 4 \mid N(1,2) = 2) = P(N(5,7) = 4)$$

Con $\mu = 10$:
$$P(N(5,7) = 4) = \frac{10^4 e^{-10}}{4!} = \frac{10000 e^{-10}}{24}$$

$$\boxed{P = 0.01891664}$$

\end{document}
