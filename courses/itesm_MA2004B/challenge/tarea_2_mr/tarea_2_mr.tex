%! Author: Facundo Bautista Barbera
%! Date: 2025-09-29
%! TEX root = tarea_2_mr.tex

\documentclass{article}
\usepackage{enumitem}

\title{Tarea 2 MR: Reto Etapa 1}
\author{
	Facundo Bautista Barbera - A01066843 \\
	Author Name - A01234567 \\
	Author Name - A01234567 \\
	Author Name - A01234567 \\
	Author Name - A01234567
}
\date{\today}

% Configurar espaciado de listas
\setlist[itemize]{itemsep=2pt, parsep=0pt, topsep=5pt}
\setlist[enumerate]{itemsep=2pt, parsep=0pt, topsep=5pt}

\newenvironment{question}[1]
{\par\vspace{10pt}\noindent\textbf{#1}\par\noindent\vspace{3pt}}
{\par\vspace{5pt}}

\begin{document}
\maketitle

\section*{Preguntas iniciales del reto}

\begin{question}{¿Qué es la reforestación?}
	La reforestación o la siembra de árboles, se refiere al conjunto de actividades que comprende la planeación, operación, el control, y la supervisión de todos los procesos involucrados en la plantación de árboles.
\end{question}

\begin{question}{¿Cuál es el objetivo de reforestar?}
	La repoblación de zonas deforestadas para recuperar bosques destruidos en el pasado reciente o antiguo para la restauración de la pérdida de biodiversidad, la reducción del dióxido de carbono en el aire,etc.
\end{question}

\begin{question}{¿Cuáles son los tipos de reforestación?}
	De acuerdo con el espacio en el que se lleva a cabo, la reforestación puede clasificarse en dos grandes categorías:

	\begin{enumerate}
		\item \textbf{Reforestación urbana}

		      Consiste en la siembra de árboles dentro de ciudades y áreas metropolitanas. Su finalidad responde a necesidades propias de la vida urbana: regular la temperatura ---ya que las áreas verdes ayudan a mitigar el calor---, mejorar la calidad del aire frente a los altos niveles de CO$_2$ generados por el tráfico, crear zonas de sombra y, al mismo tiempo, embellecer el paisaje urbano.

		\item \textbf{Reforestación rural}

		      Hace referencia a la plantación extensiva de árboles en terrenos forestales degradados o deforestados, es decir, en lugares donde anteriormente existían bosques, selvas o vegetación árida o semiárida. También puede implementarse en zonas sin cobertura forestal previa; en este último caso, se habla propiamente de forestación. Dentro de la reforestación rural se distinguen varias modalidades según la finalidad: conservación, protección y restauración, agroforestal o productiva.
	\end{enumerate}
\end{question}

\begin{question}{¿Cuál es la densidad de planta por ha de acuerdo con el tipo de vegetación?}
\end{question}

\begin{question}{¿Cuáles problemas puede generar una mala logística previa a la reforestación?}
\end{question}

\begin{question}{¿Cómo se realiza una reforestación?}
\end{question}

\begin{question}{¿Cuáles problemas puede generar una mala reforestación?}
	La reforestación puede ocasionar problemas como:

	\begin{itemize}
		\item En caso de que la reforestación sea impulsiva, puede ser un proceso contraproducente, perjudicando la diversidad de especies o a los cultivos agrícolas.
		\item Se puede empobrecer el suelo por exceso de concentración salina.
		\item Una mala elección de los nuevos árboles, así como su manera de plantarlos y posicionarlos puede ser perjudicial.
		\item La introducción de especies invasoras puede ocasionar la extinción de otras.
	\end{itemize}
\end{question}

\begin{question}{¿Qué son las plantas invasoras y qué daños pueden ocasionar?}

	Las especies invasoras de plantas son aquellas que no pertenecen al ecosistema y usualmente causan un impacto negativo ecológico, económico y social. En muchos casos estas especies son catalogadas invasoras por que carecen de sus enemigos (depredadores, patógenos, competencia) naturales.

	Pueden ocasionar:

	\begin{itemize}
		\item Reducción de la biodiversidad
		\item Alteración de procesos ecológicos
		\item Efectos hídricos y suelo
		\item Competencia difícil de revertir
		\item Impactos económicos
		\item Efectos en servicios ecosistémicos
		\item Riesgo de hibridación
	\end{itemize}
\end{question}

\begin{question}{¿Qué es un monocultivo y qué daños puede ocasionar?}
	Un monocultivo es la práctica agrícola o forestal que consiste en cultivar una sola especie de planta en un área determinada de manera continua. Aunque puede resultar más eficiente económicamente a corto plazo, presenta diversos daños ambientales y ecológicos:

	\begin{itemize}
		\item Empobrecimiento del suelo al extraer siempre los mismos nutrientes
		\item Mayor vulnerabilidad a plagas y enfermedades específicas de la especie
		\item Reducción de la biodiversidad local
		\item Dependencia excesiva de fertilizantes y pesticidas químicos
		\item Alteración de los ciclos naturales del ecosistema
		\item Erosión del suelo por falta de diversidad de raíces
		\item Mayor riesgo de pérdida total de la cosecha ante eventos adversos
	\end{itemize}
\end{question}

\begin{question}{Investigar qué es el método de Monte Carlo y en dónde se aplica y qué condiciones se requieren para utilizarlo.}
\end{question}

\begin{question}{¿Cuántas corridas se tienen que hacer para lograr cierto nivel de eficiencia?}
\end{question}

\begin{question}{¿Qué ventajas y desventajas tiene el método?}
\end{question}

\begin{question}{¿Qué es una cadena de markov?}
\end{question}

\begin{question}{¿Para qué se utiliza una cadena markov?}
\end{question}

\end{document}
