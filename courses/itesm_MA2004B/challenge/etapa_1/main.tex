% !TEX program = pdflatex
% !TEX options = -aux-directory=out -output-directory=out -synctex=1 -interaction=nonstopmode -file-line-error "%DOC"
\documentclass{article}

\usepackage[utf8]{inputenc}
\usepackage{amsmath}
\usepackage{url}
\usepackage[spanish]{babel}
\usepackage[round]{natbib}

\title{Title}
\author{
	Facundo Bautista Barbera - A01066843 \\
	Author Name - A01234567 \\
	Author Name - A01234567 \\
	Author Name - A01234567 \\
	Author Name - A01234567
}
\date{\today}

\begin{document}
\maketitle

\section*{Introducción}


A pesar de los esfuerzos globales por mitigar el cambio climático, la deforestación en México continúa siendo uno de los principales factors que agravan el deterioro ambiental.
Las actividades humanas, como el cambio de uso de suelo, la ganadería extensiva y la expansión urbana, han provocado la pérdida de miles de hectáreas de bosques y selvas, reduciendo drásticamente los servicios eco-sistémicos que estos proveen, entre ellos, la regulación del clima, la captación de agua, la producción de oxígeno y la conservación de la biodiversidad.


Una de las estrategias más importantes para revertir estos impactos es la reforestación, sin embargo, este proceso no siempre se realiza de manera adecuada.
Cuando la planeación es deficiente, pueden surgir problemas graves como la introducción de especies invasoras, la competencia por recursos entre plantas mal distribuidas, o incluso la muerte de ejemplares por falta de agua o nutrientes.
Además, una mala estimación de la cantidad de plantas necesarias genera pérdidas económicas, ya sea por excedentes no utilizados o por déficit que impiden cumplir los objetivos de restauración.

La reforestación es un proceso integral que va más allá de la simple plantación de árboles. Inicia con una planeación técnica que considera las condiciones ecológicas del área a intervenir, como el tipo de suelo, la disponibilidad de agua, el clima, la topografía y las especies nativas que se adaptan mejor al entorno.

Una mala logística previa a la reforestación puede comprometer gravemente el éxito del proyecto y generar efectos contrarios a los esperados.
La falta de planeación en la selección de especies puede provocar la introducción de ejemplares no nativos o invasores, que alteran el equilibrio ecológico y desplazan a la vegetación original.
Asimismo, una mala distribución espacial o un cálculo incorrecto del número de plantas puede ocasionar competencia excesiva por recursos, bajo crecimiento o mortalidad temprana.
Otros problemas relacionados son la reducción de biodiversidad y acelerar la degradación del suelo.
Estos impactos pueden ocasionar una pérdida de equilibrio ecológico y afectar la productividad del terreno \citep{fao2015}

Además, otro problema frecuente son los monocultivos, ya que al plantar masivamente una sola especie vegetal se reduce la diversidad genética y aumenta la vulnerabilidad ante plagas, enfermedades y sequías.
Aunque a corto plazo pueda parecer una estrategia eficiente, a largo plazo muestra muchos problemas. Además disminuyen la capacidad del ecosistema para regenerarse naturalmente y adaptarse a cambios climáticos, generando un ecosistema frágil ante las perturbaciones externas \citep{gann2019}

La decisión sobre qué especies plantar en un área de reforestación depende de factores como el tipo de suelo, el clima, altitud, disponibilidad de agua y condiciones de luz.
Por esto se prioriza el uso de especies endémicas o nativas las cuales presentan una mejor adaptación y así puedan ayudar a lograr un balance ecológico.
La distribución de las especies dentro del terreno debe planificarse cuidadosamente para minimizar la competencia por recursos y maximizar la supervivencia de las plantas.
Para esto se pueden aplicar criterios de espaciamiento y técnicas de mezcla que favorezcan la diversidad.


\section*{Trabajo Relacionado}

\textbf{Optimización de Cadena de Suministro Láctea con Consolidación Estocástica.}
La aplicación de optimización estocástica en procesos de consolidación ha demostrado resultados significativos en diversos sectores industriales. En la industria láctea argentina, una corporación cooperativa utilizó optimización estocástica para planificar flujos de materias primas y productos tras consolidar sus operaciones en 3 plantas principales, considerando variabilidad en demanda y suministro. El modelo logró coordinar eficientemente actividades logísticas y de producción en toda la cadena de suministro \citep{dairy_optimization2021}.

\textbf{Consolidación de Plantas de Manufactura en Industria de Construcción.}
En el sector de manufactura de materiales de construcción, un fabricante líder desarrolló un plan de consolidación de tres plantas de techos metálicos en una sola instalación mediante análisis de desempeño actual y evaluación de riesgos. La consolidación resultó en una reducción de costos operativos del 10\% en estado estable, con potencial adicional de eficiencias operacionales, demostrando la factibilidad operacional y atractivo financiero de estrategias de consolidación estructuradas \citep{lek2018}.

\textbf{Reducción de Costos Indirectos mediante Consolidación y Tecnología.}
El uso de tecnologías avanzadas para reducción de complejidad también ha mostrado impactos cuantificables. Un estudio de 24 empresas industriales implementó automatización, inteligencia artificial y análisis avanzado para consolidar proveedores y optimizar costos indirectos en áreas como finanzas, procurement y recursos humanos. Las empresas lograron reducciones de costos entre 15-20\% en 12-18 meses mediante consolidación efectiva y optimización de políticas de gasto \citep{mckinsey2020}.

\textbf{Consolidación de Partes en Manufactura Aditiva.}
En manufactura aditiva metálica, la optimización concurrente de consolidación de partes y parámetros de proceso ha demostrado beneficios significativos. Un estudio logró reducir de 19 a 7 partes con 20\% menos peso, resultando en 15\% de reducción en costos totales de producción y 21\% de reducción en tiempo de construcción. Esta consolidación también simplificó la gestión al reducir el número de partes a rastrear, adquirir e inspeccionar \citep{additive_manufacturing2019}.

\textbf{Simplificación de Tareas mediante Lean Six Sigma.}
Las metodologías Lean Six Sigma han sido aplicadas exitosamente para simplificar procesos mediante manufactura celular, estandarización y consolidación de operaciones. La reducción de complejidad en competencias y tareas resultó en ambientes de trabajo simplificados que facilitan el aprendizaje de procedimientos, incrementan la flexibilidad de la fuerza laboral y reducen errores de manufactura, mejorando la eficiencia general del sistema productivo \citep{lean_six_sigma2025}.

\subsection*{Coloración de Arcos y Aristas}

La coloración de aristas es una asignación de "colores" a las aristas de un grafo de manera que dos aristas incidentes (que comparten un vértice) no tengan el mismo color. El número mínimo de colores necesarios se denomina \textbf{índice cromático} $\chi'(G)$. En grafos dirigidos, las aristas tienen dirección y se denominan arcos.

El \textbf{Teorema de Vizing} establece que para cualquier grafo simple, el índice cromático está entre $\Delta(G)$ y $\Delta(G)+1$, donde $\Delta$ es el grado máximo del grafo. Determinar si un grafo necesita $\Delta$ o $\Delta+1$ colores es un problema NP-completo.

\textbf{Aplicaciones en telecomunicaciones y redes.} En este dominio, la coloración de aristas se utiliza para asignación de frecuencias evitando interferencias entre transmisores adyacentes en redes celulares, optimización de longitudes de onda en redes de fibra óptica, y programación eficiente en switches de alta velocidad mediante conmutación de paquetes.

\textbf{Aplicaciones en planificación y horarios.} La coloración de aristas resuelve problemas de programación de torneos deportivos mediante esquemas round-robin donde cada equipo juega una vez con todos los demás. También se aplica en la generación de horarios universitarios, donde los cursos son vértices y los conflictos de tiempo son aristas, así como en la programación de exámenes para evitar que estudiantes tengan exámenes simultáneos.

\textbf{Aplicaciones en computación.} Los compiladores utilizan coloración de aristas para asignación de registros, asignando variables a registros del procesador. En computación paralela, permite la programación de tareas sin conflictos de recursos.

Los algoritmos principales para resolver este problema incluyen el algoritmo voraz, que asigna secuencialmente el menor color disponible a cada arista; el algoritmo de backtracking, que explora todas las coloraciones posibles retrocediendo ante conflictos; y el algoritmo de Vizing, que garantiza encontrar una coloración con máximo $\Delta+1$ colores.

\section*{Contexto del problema}

\section*{Definición del problema}

\newpage
\nocite{*}
\bibliographystyle{apalike}
\bibliography{references}

\end{document}
